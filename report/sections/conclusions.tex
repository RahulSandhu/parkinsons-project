\section{Conclusions}

The purpose of this study was to detect Parkinson’s disease (PD) using voice
measurements and to evaluate the performance of three different preprocessing
approaches with a KNN model: the raw preprocessed model, the averaged model,
and the normalized model. Exploratory data analysis was performed to identify
outliers, missing values, and inconsistencies in the raw data. The dataset was
then preprocessed through renaming, outlier handling, aggregation, and
normalization. Dimensionality reduction and correlation-based feature selection
were applied to optimize model performance.

The results demonstrated that preprocessing methods significantly influence the
accuracy and stability of the KNN model. Among the three approaches, the
normalized model (\texttt{df\_norm}) provided the best performance, achieving
an optimal \(k = 4\) with nearly 97\% accuracy, compared to approximately 83\%
accuracy with the raw preprocessed model (\texttt{df\_clean}). These findings
underscore the importance of normalizing and scaling features to ensure
compatibility and improve model performance. The averaged model
(\texttt{df\_avg}) failed to achieve convergence, likely due to insufficient
data points and the resulting loss of critical information. This suggests that
averaging may only be effective on substantially larger datasets.

The integration of the project workflow into an API enhanced model management,
evaluation, and prediction capabilities. The user-friendly frontend of the API
allows users to explore different model versions, upload data, and obtain
predictions seamlessly. This implementation not only facilitates collaboration
but also provides a foundation for real-world applications of the model.

However, the study has several limitations. The models were trained on a small
dataset consisting of 31 individuals, which limits the generalizability of the
results. Training on larger datasets would produce a more robust predictive
model. Additionally, outlier handling, which replaced outlier values with
within-subject non-outlier averages, may have excluded important
Parkinson’s-specific diagnostic information, potentially reducing the model’s
sensitivity. Dimensionality reduction and feature selection were based solely
on correlation metrics, which capture only linear relationships and may
overlook non-linear interactions between features. Finally, the model is
designed to classify data into a binary outcome (Parkinson’s or no
Parkinson’s), whereas predicting disease stages or progression would provide
greater clinical value.

Future directions for this work include expanding the dataset to improve model
robustness, integrating additional biomarkers (e.g., handwriting or gait
analysis) alongside voice data, and enhancing the API to support real-time data
input and predictions. These advancements would significantly increase the
utility of the system, particularly in clinical settings, where early detection
and personalized treatment of PD are critical.
